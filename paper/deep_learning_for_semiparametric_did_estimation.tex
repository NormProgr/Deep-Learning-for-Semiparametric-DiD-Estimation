\documentclass[11pt, a4paper, leqno]{article}
\usepackage{a4wide}
\usepackage[T1]{fontenc}
\usepackage[utf8]{inputenc}
\usepackage{float, afterpage, rotating, graphicx}
\usepackage{epstopdf}
\usepackage{longtable, booktabs, tabularx}
\usepackage{fancyvrb, moreverb, relsize}
\usepackage{eurosym, calc}
% \usepackage{chngcntr}
\usepackage{amsmath, amssymb, amsfonts, amsthm, bm}
\usepackage{caption}
\usepackage{mdwlist}
\usepackage{xfrac}
\usepackage{setspace}
\usepackage[dvipsnames]{xcolor}
\usepackage{abstract}
\usepackage{subcaption}
\usepackage{minibox}
% \usepackage{pdf14} % Enable for Manuscriptcentral -- can't handle pdf 1.5
% \usepackage{endfloat} % Enable to move tables / figures to the end. Useful for some
% submissions.
\usepackage{threeparttable}
\usepackage{multicol}
\usepackage[
    natbib=true,
    bibencoding=inputenc,
    bibstyle=authoryear-ibid,
    citestyle=authoryear-comp,
    maxcitenames=3,
    maxbibnames=10,
    useprefix=false,
    sortcites=true,
    backend=biber,
    doi=false,
    url=false,
    isbn=false,
]{biblatex}

\AtEveryBibitem{
    \clearfield{issn}
    \clearfield{note}
}


\AtBeginDocument{\toggletrue{blx@useprefix}}
\AtBeginBibliography{\togglefalse{blx@useprefix}}
\setlength{\bibitemsep}{1.5ex}
\addbibresource{refs.bib}
\graphicspath{ {./graphs/} }
\usepackage[unicode=true]{hyperref}
\hypersetup{
    colorlinks=true,
    linkcolor=black,
    anchorcolor=black,
    citecolor=NavyBlue,
    filecolor=black,
    menucolor=black,
    runcolor=black,
    urlcolor=NavyBlue
}


\widowpenalty=10000
\clubpenalty=10000

\setlength{\parskip}{1ex}
%\setlength{\parindent}{0ex}
\setstretch{1.5}

\usepackage{acronym}
\acrodef{iid}[i.i.d.]{\textbf{independent and identically distributed}}
\acrodef{did}[DiD]{\textbf{Difference-in-Differences}}
\acrodef{ols}[OLS]{\textbf{Ordinary Least Squares}}
\acrodef{twfe}[TWFE]{\textbf{Two-Way Fixed Effects}}
\acrodef{twfer}[TWFEr]{\textbf{Two-Way Fixed Effects Regression}}
\acrodef{OWFE}[OWFE]{\textbf{One-Way Fixed Effects}}
\acrodef{mcs}[MCS]{\textbf{Monte Carlo Simulation}}
\acrodef{att}[ATT]{\textbf{average treatment effect on the treated}}
\acrodef{pta}[PTA]{\textbf{parallel trends assumption}}
\acrodef{ipw}[IPW]{\textbf{Inverse Probability Weighting}}
\acrodef{drdid}[DRDiD]{\textbf{Doubly-Robust DiD}}
\acrodef{or}[OR]{\textbf{Outcome Regression}}
\acrodef{dgp}[DGP]{\textbf{data generating process}}
\acrodef{mlp}[MLP]{\textbf{multilayer perceptron}}
\acrodef{relu}[ReLU]{\textbf{rectified linear unit}}


\begin{document}


\begin{titlepage}

\begin{center}

\vspace*{0.4cm}

\huge {\bfseries Deep Learning for Semiparametric Difference-in-Difference Estimation}

\vspace{1cm}

\large {Master Thesis Presented to the}\\
\large {Department of Economics at the}\\
\large {Rheinische Friedrich-Wilhelms-Universität Bonn}\\

\end{center}

\vspace{1cm}

\begin{center}


\large {In Partial Fulfillment of the Requirements for the Degree of}\\
\large {Master of Science (M.Sc.)}\\

\end{center}
\vspace{1cm}
\begin{center}

\vspace*{1cm}


\large {Supervisor: Prof. Dr. Christoph Breunig}\\

\end{center}

\vspace{1cm}

\begin{center}

\vfill


\large {Submitted in \today \, by:}\\
\large {Norman Lothar Metzinger}\\
\large {Matriculation Number: 3501090}\\

\end{center}

\vspace{1cm}



\setcounter{page}{0}\clearpage




\end{titlepage}

\endinput

%\input{./abstract.tex}
\thispagestyle{empty}
\begin{abstract}
\noindent This thesis explores the implementation of deep feedforward neural networks into semiparametric Difference-in-Differences estimation (DiD), highlighting its potential under conditional parallel trends assumption.
It reviews current classical and deep learning estimation methods for first-step DiD estimation and conducts a Monte Carlo Simulation to test their validity for second-step inference.
The results demonstrate that deep learning performs nearly as well as the best classical approaches and outperforms those in scenarios with incorrectly specified outcomes.
To further investigate deep learning, multiple deep learning architectures are tested, showing sensitivity towards their hyperparameters.
Finally, DiD deep learning estimators show promise in real-world applications, handling heterogeneous treatment effects effectively.
\end{abstract}
\clearpage
\tableofcontents
\thispagestyle{empty}
\newpage
\setcounter{page}{1}

\section{Introduction}



% Issues and importance of Inference and DIfference in Difference and how Deep Learning could tackle that

\ac{did} is a widely used econometric method to estimate the effect of a policy change on a group of individuals called treatment group.
To achieve this, the method requires to compare treatment group to a control group before and after the policy change.
The important underlying assumption is the \ac{pta}, which states that the treatment and control group would have developed similarly in the absence of the policy change.
This assumption is key to identify the effect on the treatment group, the \ac{att}, to be causal.

However in practice one does not know if the \ac{pta} holds as it is by design untestable.
If individuals are selected into treatment based on characteristics that also influence the outcome, the \ac{pta} is violated.
To overcome this issue researchers condition on these characteristics such that they assume conditional \ac{pta} \citep[see][]{santannaDoublyRobustDifferenceindifferences2020,manfeDifferenceInDifferenceDesignRepeated}

In this thesis I want to adress how researchers can use a more flexible semiparametric approach to achieve robust \ac{did} estimation under conditional \ac{pta}.
For this I consider a variety of machine and deep learning models to interchange the first-step in the \ac{did} estimation.
Especially the use of deep learning marks a novelty and I follow \citet{farrellDeepNeuralNetworks2021} to contribute to the young but growing literature.
% Importance of Deep Learning for Inference and for Economics



% say what the exact finding is in the following parts and keep the structure with First, Second, ... .
%
This thesis wants to contribute in four ways to the literature.
First, I want to discuss the current state of classical and machine learning techniques used for \ac{did} estimation.
For this I revise the classic \ac{did} estimation with \ac{twfe}, then I revise the semiparametric approaches such as \ac{or} \citep[see][]{heckmanMatchingEconometricEvaluation1998}, \ac{ipw} \citep[see][]{abadieSemiparametricDifferenceinDifferencesEstimators2005}, and \ac{drdid} \citep[see][]{santannaDoublyRobustDifferenceindifferences2020}.

Second, I introduce a new approach using deep learning for first step \ac{did} estimation.
As the literature is rather new I want to revise how deep neural networks work and why they can be valid for inference following the results of \citet{farrellDeepNeuralNetworks2021}.
Third, I want to provide a comprehensive \textit{Monte Carlo Simulation} to compare the performance of these techniques. %explain more
Lastly, I want to apply these techniques to a real-world dataset to show the potential of these techniques. %explain more


% Organization of the paper
 % (fold)
\section{Methodology}




\subsection{2x2 Difference in Differences}
%derive the explanation of 2x2 difference in difference from classic to TWFE and talk about heterogeneous effects
To introduce a common ground for the rest of the thesis, I want to introduce the notation for the basic 2x2 \ac{did} model.
The model has two time periods given by $T$, where $t \in 0, 1$.
These define the pre- and post-treatment period of the policy change.
Throughout the thesis, I use panel data, with $i$ denoting the individual observed over time.
The two groups are defined by $D$, where $d \in 0, 1$, such that $d = 1$ is the treatment group and $d = 0$ is the control group.
The outcome variable is given by $Y$ and the variable of interest, the \ac{att}, is given by $\tau^{fe} = \mathbb{E}(Y_{1,1} - Y_{0,1} \mid  X,D = 1)$.
Therefore the \ac{att} describes the expected difference in outcomes between the treated group (when they receive the treatment) and what their outcomes would have been if they had not received the treatment.

For my \ac{mcs} I use the following common \ac{twfe} model notation to display the 2x2 \ac{did} as in \citet{santannaDoublyRobustDifferenceindifferences2020}:
\begin{equation}
Y_{it} = \alpha_1 + \alpha_2 T_i + \alpha_3 D_i + \tau^{fe} (T_i \cdot D_i) + \theta' X_i + \epsilon_{it}
\label{eq:twfe}
\end{equation}
Equation \ref{eq:twfe} implicates two assumptions that are of main focus in this thesis.
First, it assumes homogeneity in treatment effects, such that $\tau^{fe}$ is constant over all individuals.
Second, it assumes that the \ac{pta} holds, such that the treatment and control group would have developed similarly in the absence of the policy change such that $\mathbb{E} [Y_1 - Y_0 \mid X, D = d] = \mathbb{E} [Y_1 - Y_0 \mid D = d]$.
If one or both of these assumptions are violated the \ac{twfe} estimator in Eq. \ref{eq:twfe} is inconsistent and biased.

To control for heterogeneous treatment effects and to account for conditional \ac{pta}, we can extend the model in Eq. \ref{eq:twfe} by adding interactions of $X$, $T$, and $D$ \citep[see][]{manfeDifferenceInDifferenceDesignRepeated, hansen2022econometrics}.
Therefore, one can rewrite the Eq. \ref{eq:twfe} the following:
\begin{equation}
    Y_{it} = \alpha + \gamma T_{it} + \beta D_{i} + \tau^{corr} (T_{it} \cdot D_{i}) + X_{it}' \theta + (T_{it} \cdot X_{it}') \omega + (D_{i} \cdot X_{it}') \nu + (T_{it} \cdot D_{i} \cdot X_{it}') \rho + \epsilon_{it}
    \label{eq:twfecorr}
\end{equation}
Note that $T_{it} \cdot D_{i} \cdot X_{it}'$ is the change of the treatment depending on X, thus the conditional \ac{pta} holds \citep{manfeDifferenceInDifferenceDesignRepeated}.
Eq. \ref{eq:twfecorr} is therefore, in a correctly specified case, neither biased nor inconsistent.
The issue is that the econometric practitioner needs good reasoning and understanding to add the correct interactions.
In the following sections, I introduce more flexible techniques circumventing this issue.

\subsection{Outcome Regression}
In this part, I revise \ac{or} as it is an important technique used in \ac{drdid}, rather than using \ac{or} itself for estimating.
\ac{or} is a generalized \ac{did} estimation approach that estimates the outcomes as a function of covariates, given by $Y_i = g_i(X) + \epsilon_i$, where $i \in 0,1$.
The basic idea is to predict the control group outcomes based on their covariates and then compare these predicted outcomes to the actual outcomes observed for the treated group.
The prediction can be computed through a linear regression or other non-linear models like p-nearest neighbor matching \parencite{heckmanMatchingEconometricEvaluation1998}.

In this thesis, I fit a regression to estimate \ac{or} within the \ac{drdid} framework, allowing to formulate the following model:
\begin{equation}
\hat{\tau}^{or} = \bar{Y}_{1,1} - \bar{Y}_{1,0} - \left[ \frac{1}{n_{treat}} \sum_{i \mid D_i = 1} \left( \hat{\mu}_{0,1}(X_i) - \hat{\mu}_{0,0}(X_i) \right) \right],
\label{eq:3}
\end{equation}
where $\bar{Y}_{1,1} - \bar{Y}_{1,0}$ is the average outcome among treated units between pre- and post-treatment period.
The part in brackets of eq. \ref{eq:3} is the difference between the predicted control outcomes in the post- and the predicted control outcomes in the pre-treatment period.
The key expression is $\hat{\mu}_{d,t}(X)$ which estimates the true, unknown $m_{d,t}(x) \equiv \mathbb{E}[Y_t \mid D = d, X = x]$.
Intuitively, it estimates what the outcome would be for a person with specific traits if they were either treated or not treated.
Note that if $\hat{\mu}_{d,t}(X)$ is linear, it would be close to the correct \ac{twfe} estimator $\tau^{corr}$.
Therefore, it is crucial that $\hat{\mu}_{d,t}(X)$ is correctly specified; otherwise, the \ac{att} is biased and inconsistent.

\subsection{Inverse Probability Weighting}
The \ac{ipw} estimator is another common approach to estimate \ac{att}, which relaxes the conditional \ac{pta} as considered in this thesis.
Contrary to the \ac{or} approach, \ac{ipw} does not directly model the change in the outcome \citep{santannaDoublyRobustDifferenceindifferences2020}.
Instead, the idea is to only control for covariates that affect the probability of the treatment.
If the probability of an individual receiving the treatment or being in the control group is the same, then the only difference between control and treatment is chance.
Thus, there are no biases through confounding variables.

Therefore, it is crucial to correctly estimate the probability of being treated \citep{angrist2009mostly}.
The true probability is estimated by the so-called propensity score, given by $p(x) = P(D=1 \mid X)$, which is not directly observable.
Therefore it is estimated by $\hat{\pi}(X)$.
Note that there are several ways to estimate the propensity score, such as logistic regression, probit regression, or machine learning techniques.
These techniques are used in the first step to estimate the propensity score, and in the second step, the outcome model is estimated parametrically \citep{abadieSemiparametricDifferenceinDifferencesEstimators2005}.
In this thesis, I use logistic regression and deep neural networks to estimate the propensity scores.

The \ac{ipw} estimator is given by:
\begin{equation}
\hat{\tau}^{\text{ipw}} = \frac{1}{\mathbb{E}_n[D]} \mathbb{E}_n \left[ \frac{D - \hat{\pi}(X)}{1 - \hat{\pi}(X)} (Y_1 - Y_0) \right],
\label{eq:4}
\end{equation}
where $\mathbb{E}_n$  is the sample average of the treatment $D$.
The term $\frac{D - \hat{\pi}(X)}{1 - \hat{\pi}(X)}$ reweights the treatment and control to account for the probability of receiving the treatment.
$Y_1 - Y_0$ captures the change in the outcome for each individual.

Lastly, there are two remarks regarding the \ac{ipw} estimator.
First, the \ac{ipw} estimator is consistent and unbiased if the propensity scores are correctly specified.
Second, it is crucial to consider all relevant covariates in the propensity score estimation \citep{angrist2009mostly}, than to improve the prediction of propensity scores \citep{https://doi.org/10.3982/ECTA18515}.
The reason is that including irrelevant covariates to improve the prediction of the propensity scores also increases the variance of the estimator, without adding any more information \citep{hernanCausalInferenceWhat}.

\subsection{Double Robust Difference in Differences}

The \ac{drdid} of \citet{santannaDoublyRobustDifferenceindifferences2020} is a combination of the two approaches discussed before; \ac{or} and \ac{ipw}.
The \ac{drdid} identifies the \ac{att} correctly if either the \ac{or} or \ac{ipw} is correctly specified.
In this case, the aforementioned weaknesses of either approach are avoided which makes the \ac{drdid} double robust.

Recall from before that $\hat{\pi}(X)$ estimates $p(X)$ the true, unknown propensity score model. $\mu_{d,t}(X)$ is a model for the true, unknown outcome regression $m_{d,t}(x)\equiv \mathbb{E}[Y_t | D = d, X = x]$, $d, t = 0, 1$.
In this thesis, I only view panel data such that I can write $\Delta Y = Y_1 - Y_0$ for the change in the outcome.
The expression $\mu_{d,\Delta}(X) \equiv \mu_{d,1}(X) - \mu_{d,0}(X)$ represents the difference in the expected outcomes before and after treatment, adjusted for covariates $X$, for the group with treatment status $D=d$.

Thus, one can se how \ac{or} and \ac{ipw} are constructed within the \ac{drdid} estimator, given by:
\begin{equation}
\tau^{dr} = \mathbb{E} \left[ \left( w_1(D) - w_0(D, X; \hat{\pi}) \right) \left( \Delta Y - \mu_{0, \Delta}(X) \right) \right],
\label{eq:5}
\end{equation}
where $w_1(D) - w_0(D, X; \hat{\pi})$ directly corresponds to the \ac{ipw} estimator and $\Delta Y - \mu_{0, \Delta}(X)$ corresponds to the \ac{or} estimator from eq. \ref{eq:3} and eq. \ref{eq:4} respectively.
Note that $w_1(D)$ is a weighting assigned to the treatment group and $w_0(D, X; \hat{\pi})$ is a weighting assigned to the control group, are given by:
%explain that more and check if it is correctly written to my notation
\begin{equation}
w_1(D) = \frac{D}{\mathbb{E}[D]}, \quad  \mathit{and} \quad
w_0(D, X; \hat{\pi}) = \frac{\hat{\pi}(X)(1-D)}{1 - \hat{\pi}(X)} \Bigg/ \mathbb{E} \left[ \frac{\hat{\pi}(X)(1-D)}{1 - \hat{\pi}(X)} \right].
\label{eq:6}
\end{equation}
The \ac{drdid} estimator is consistent and unbiased if both \ac{or} and \ac{ipw} are correctly specified but it is less obvious if only one of the two is correctly specified.
To clarify this, assume the \ac{ipw} is incorrectly specified and the \ac{or} is correctly specified.
The incorrect specification of \ac{ipw} is reflected in $w_0(D, X; \hat{\pi})$ in eq. \ref{eq:6} because $\hat{\pi}$ is biased.
Meaning the weight for the control group is misspecified for \ac{ipw}.
This effect is nullified by the correct specification of \ac{or} in $\Delta Y - \mu_{0, \Delta}(X)$ because the change in the outcome evolution is zero in expectation.
Intuitively, the \ac{or} correctly identifies that the change in the outcome of control should not change over time, as it is not treated, therefore any multiplication of it becomes zero as well.
A similar argument can be made for the \ac{or} being misspecified and the \ac{ipw} being correctly specified.

\section{Deep Learning for Inference}

\subsection{Revision of Deep Learning}
Deep learning is a rapidly developing field within machine learning that recently got a lot of attention within economics.
The idea is to take complex data and to represent it by a series of simpler representations, each of which is expressed in terms of the previous one \citep{Goodfellow-et-al-2016}.
A common example of this is the \textit{feedforward neural network}, which is a series of layers of neurons, each of which is connected to the next layer.
The first layer is the input layer, the last layer is the output layer, and the layers in between are called hidden layers.
The input layer corresponds to the covariates $X$, the output layer corresponds to the outcome $Y$.
Figure \ref{fig:1} illustrates the layer and node structure of a \ac{mlp}, which is a special class of feed forward networks and is commonly used in empirical applications \citep{farrellDeepNeuralNetworks2021}.
From hence on I use \ac{mlp} and deep learning interchangeably as it is the approach used in this paper.

\begin{figure}%                 use [hb] only if necceccary!
\centering
\caption{Illustration of a feedforward neural network \citep{farrellDeepNeuralNetworks2021}}
\includegraphics[width=\textwidth]{Neural_net}
\caption*{\textbf{Note:} This figure illustrates the basic structure of a \ac{mlp} $\mathcal{F}_{\text{MLP}}$, showing the input layer with $d=2$ neurons in white. The two ($H=2$) hidden layers in grey with $U=6$ neurons, and one output layer in black ($L=1$). The total amount of weights is $W=28$.}
\label{fig:1}
\end{figure}

The actual computation within the neural networks is done by the \textit{activation function} $ \sigma : \mathbb{R} \to \mathbb{R} $, which is applied to the output of each hidden neuron.
The most common activation function is the \ac{relu} function, which is defined as $ \sigma(x) = \max(0, x) $ and used in this thesis.
The advantage of the linear \ac{relu} is that it is computationally fast and the circumvents the vanishing gradient problem, which is a common problem in deep learning.\footnote[1]{The vanishing gradient issue arises especially by activation functions like \textit{sigmoid} and \textit{tanh}.
When the neural network model is trained, all the weights of the model are updated through a process called \textit{backpropagation}.
Backpropagation is the algorithm used to compute the gradient of the loss function with respect to each parameter, which is then used to update the parameters in the direction that minimizes the loss.
The issue that can arise is that updating of parameters is hindered or training is completely stopped \citep{abuqaddom2021oriented}.}
\ac{relu} takes any linear combination given by $\tilde{x}' w + b$ and transforms it to $ \sigma(\tilde{x}' w + b)$, where $w$ is the weight vector, $b$ is the constant term\footnote[2]{The actual term in computer science is \textit{bias} but to reduce confusion with the inference use of the term I follow \citet{farrellDeepNeuralNetworks2021} and use the term \textit{constant} as well.}, and $\tilde{x}$ is the input vector.
One can clearly see that the \ac{relu} sets all negative values from the linear combination $\tilde{x}' w + b$ to zero, while keeping all positive values unchanged.
%dont do the footnote write that out here

The main problem the neural network wants to solve is to estimate the unknown function $f^*(x)$.
More precisely, $f^*$ is a function that maps the input $\tilde{x}$ to the output $\tilde{y}$.
As $f^*$ is unknown, the neural network tries to estimate it by minimizing the expected loss function $\mathbb{E}[\ell(f, Z)]$, such that one can write:
\begin{equation}
f^* = \arg \min_f \mathbb{E}[\ell(f, Z)],
\end{equation}
where $\ell(f, Z)$ is the loss function, which is a measure of how well the model is performing. %change it here
The loss function can be represented by many different function such as least squares or logistic regression, where the latter one is used in this thesis.
\begin{equation}
f^*(x) := \log \left( \frac{\mathbb{E}[Y|X = x]}{1 - \mathbb{E}[Y|X = x]} \right) \quad \text{and} \quad \ell(f, z) = -yf(x) + \log(1 + e^{f(x)}). \quad \text{(2.3)}
\end{equation}

\subsection{Deep Learning for Difference in Differences}

\section{Monte Carlo Simulations}

\begin{table}[htbp]
\centering
\resizebox{\linewidth}{!}{
\begin{threeparttable}
\caption{Monte Carlo Simulation with Homogenous Treatment Effects}
\label{tab:table1}
\begin{tabular}{lllllll}
\toprule
\hline
\addlinespace
Estimator         & Reference                         & Av. Bias   & Med. Bias   & RMSE & Variance & Cover \\ \midrule
\addlinespace
\large \textbf{DGP1}            &                                   &            &             &      &           \\
\addlinespace
$\hat{\tau}^{fe}$ & Regression, Eq. \eqref{eq:twfe}               & -20.963       & -20.816        & 21.277 & 13.247& 0.000      \\
$\hat{\tau}^{corr}$ & Regression, Eq. (2)             & -0.002       & -0.001        & 0.196 & 0.038  &  0.840   \\
$\hat{\tau}^{ipw}$ & Abadie (2005)                    & -0.376       & -0.469        & 9.396 & 45.704 &  0.840      \\
$\hat{\tau}^{ipw,dl}$ & Abadie (2005) + DL            & -3.819       & -3.697        & 3.841 & 36.065 &      \\
$\hat{\tau}^{dr}$ & Sant'Anna and Zhao (2020)         & 0.003      & 0.008       & 0.218 & 0.022  &  0.834   \\
$\hat{\tau}^{dr,dl}$ & Sant'Anna and Zhao (2020) + DL & -0.121       & -0.120        & 0.121 & 0.020  &     \\\midrule


\addlinespace
\large \textbf{DGP2}            &                                   &            &             &      &           \\
\addlinespace
$\hat{\tau}^{fe}$ & Regression, Eq. \eqref{eq:twfe}               & -19.261      & -19.040      & 19.606 & 13.403 & 0.000      \\
$\hat{\tau}^{corr}$ & Regression, Eq. (2)             & -0.004       & -0.001        & 0.195 & 0.038  & 0.832     \\
$\hat{\tau}^{ipw}$ & Abadie (2005)                    & -0.498       & -0.472        & 9.660 & 47.106  & 0.839    \\
$\hat{\tau}^{ipw,dl}$ & Abadie (2005) + DL            & -21.983       & -22.114        & 22.335 & 43.872     \\
$\hat{\tau}^{dr}$ & Sant'Anna and Zhao (2020)         & 0.005       & 0.002        & 0.207 & 0.021  & 0.802    \\
$\hat{\tau}^{dr,dl}$ & Sant'Anna and Zhao (2020) + DL  & -0.148       & -0.150        & 0.148 & 0.020        \\  \midrule


\addlinespace
\large \textbf{DGP3}            &                                   &            &             &      &           \\
\addlinespace
$\hat{\tau}^{fe}$ & Regression, Eq. \eqref{eq:twfe}               & 13.122       & 12.899       & 14.028 & 24.575 & 0.109    \\
$\hat{\tau}^{corr}$ & Regression, Eq. (2)             & 0.142       & -0.114       & 4.869 & 23.685 &0.782   \\
$\hat{\tau}^{ipw}$ & Abadie (2005)                    & 0.109     & 0.219      & 9.630 & 43.498  & 0.817   \\
$\hat{\tau}^{ipw,dl}$ & Abadie (2005) + DL            & -0.810       & -0.794        & 0.824 & 40.227       \\
$\hat{\tau}^{dr}$ & Sant'Anna and Zhao (2020)         & -0.104       & 0.052        & 4.599 & 11.165  &0.840    \\
$\hat{\tau}^{dr,dl}$ & Sant'Anna and Zhao (2020) + DL & 0.228       & 0.219        & 0.293 & 11.240      \\  \midrule


\addlinespace
\large \textbf{DGP4}            &                                   &            &             &      &           \\
\addlinespace
$\hat{\tau}^{fe}$ & Regression, Eq. \eqref{eq:twfe}              & -16.434       & -16.283        & 17.226 & 26.633  & 0.033   \\
$\hat{\tau}^{corr}$ & Regression, Eq. (2)             & -3.063       & -3.165       & 6.162 & 28.588  &0.654   \\
$\hat{\tau}^{ipw}$ & Abadie (2005)                    & -3.881       & -4.063        & 10.576 & 47.230    & 0.798  \\
$\hat{\tau}^{ipw,dl}$ & Abadie (2005) + DL            & -4.992       & -4.962        & 5.005 & 43.947      \\
$\hat{\tau}^{dr}$ & Sant'Anna and Zhao (2020)         &-3.177      &-3.162       & 5.899 & 12.259  &0.752    \\
$\hat{\tau}^{dr,dl}$ & Sant'Anna and Zhao (2020) + DL & 1.630       & 1.593        & 1.652 & 15.876      \\


\bottomrule
\end{tabular}
\vspace{1em}
\begin{tablenotes}
\item Notes: Simulations based on panel data with sample size $n = 1000$ and 1000 Monte Carlo repetitions. The average bias "Av. Bias", median bias "Med. Bias", root mean squared error "RMSE", and average variance "Variance" of the estimators are reported. The "Cover" describes the coverage probability of how often the treatment coefficient falls within the confidence intervall. The methods that predict propensity scores with deep learning are marked by "DL". The true treatment effect is $\tau = 0$ in all cases and homogenous.
\end{tablenotes}
\end{threeparttable}}
\end{table}





% Please add the following required packages to your document preamble:
% \usepackage{booktabs}
% Please add the following required packages to your document preamble:
% \usepackage{booktabs}
\begin{table}[]
\centering
\caption{Performance of the Neural Network across DGPs}
\label{tab:table2}
\begin{tabular}{@{}lllll@{}}
\toprule
\addlinespace
Minimum Loss                & DGP1 & DGP2 & DGP3 & DGP4 \\ \midrule
Training    & data & data & data & data \\
Validation  & data & data & data & data \\ \bottomrule
\end{tabular}
\end{table}


\section{Application}

An early application of \ac{did} is the paper of \citet{meyer1990workers} who investigate the effect of a workers' compensation reform on their time out of work.
In 1980, the states of Kentucky and Michigan substantially increased the compensations in case of work-induced disability or injury.
As the policy affected high-earning workers, \citet{meyer1990workers} took low-earning workers as a control group.
Their idea was that low- and high-earning workers are comparable except that high-earning workers are treated with the compensation policy.
The distribution of the pre-treatment out-of-work duration for low- and high-earning workers can be seen in the Appendix Figure \ref{fig:log_duration_distribution}.
In their original study, they report a significant increase in time out of work for Kentucky but not for Michigan.

\citet{meyer1990workers} implemented a classical $2 \times 2$ \ac{did} design, which makes it suitable to the methods discussed in this thesis.
Due to the low sample size of the Michigan data, the analysis is solely focused on Kentucky.
Therefore, the \ac{did} identification strategy can be formulated as follows:
\begin{equation}
\text{Duration}_{it} = \alpha + \beta_1 \text{Post}_t + \beta_2 \text{HighEarnings}_i + \beta_3 (\text{Post}_t \times \text{HighEarnings}_i) + \gamma X_{it} + \epsilon_{it},
\label{eq:duration}
\end{equation}
where the interaction $(\text{Post}_t \times \text{HighEarnings}_i)$ is the \ac{did} estimator.
$X_{it}$ is a vector of control variables such as injury type, age, or gender.
As \citet{meyer1990workers} use many of these pre-treatment covariates like age or gender it implicates that they assume conditional \ac{pta}.
By design, it is not possible to test for \ac{pta} but the conditional \ac{pta} seems to be a more robust assumption in an observational study \citep{santannaDoublyRobustDifferenceindifferences2020}.
A second remark is towards heterogeneity in the treatment group, which is given in almost all contexts \citep{DeepLearningIndividual2021}.
The Appendix Table \ref{tab:duration} shows the difference in duration of out-of-work time across injury types before and after the treatment.
The magnitude of the differences is quite large, hinting towards heterogeneity in the treatment group.
Based on that, I conducted a regression exclusion test following \citet{hansen2022econometrics} where the results can be seen in Table \ref{tab:exclusion_test} in the Appendix.
The regression exclusion test is significant, which implies that homogeneous treatment effects cannot be assumed.

These results are reason to apply the \ac{drdid} + DL\footnote[5]{Here I use the prebuilt software implementation of \citet*{doubleml2024R} to be able to report summary statistics for the \ac{drdid} + DL estimation.} estimator to the data of \citet{meyer1990workers}.
To compare different designs, I also estimate a saturated dummy design without controls and the regression equation \ref{eq:duration} with controls and interactions.
The estimates are reported in Table \ref{tab:reg_results}.
Note that the results of the regression equation \ref{eq:duration} are slightly different from the results of \citet{meyer1990workers}.
This should be due to different handling of the data-cleaning process than to the different estimation methods.

\begin{table}[ht]
\centering
\begin{threeparttable}
\caption{Regression Results}
\label{tab:reg_results}
\begin{tabular}{lcccccc}
\toprule
\hline
\addlinespace
Model & Coef. & Std.Err. & t & P>|t| & [0.025 & 0.975] \\
\midrule
Saturated design & 0.191 & 0.069 & 2.782 & 0.005 & 0.056 & 0.325 \\
Regression Eq. (14)  & 0.172 & 0.064 & 2.694 & 0.007 & 0.047 & 0.297 \\
\ac{drdid} + DL & 0.250 & 0.075 & 3.34 & 0.000 & 0.103 & 0.396\\
Authors' model & 0.162 & 0.059 &  2.745 & 0.006 & 0.046 & 0.278 \\
\bottomrule
\end{tabular}
\begin{tablenotes}
    \item Notes: In this table are reported the results of a saturated dummy design without controls, the regression equation (14) with controls and interactions, the \ac{drdid} with deep learning and controls and the results from \citet{meyer1990workers} of Table 6 Column (ii). Reported are the coefficients as the \ac{att}, the standard errors, the t-values, P>|t| is the p-value, and the lower- and upper bound of the 95 percent confidence interval. The dependent variable is the log of the duration of work leave. The dataset is taken from the online resources of \citet{wooldridge2019introductory}. The sample size is $n = 5347$.
\end{tablenotes}
\end{threeparttable}
\end{table}


In Table \ref{tab:reg_results}, one can see that the results for all parametric models are quite similar.
Note that the results of the saturated design are similar to the regression equation \ref{eq:duration} implying that the controls might not have a large impact on the results.
Assuming a homogenous treatment effect this result would normally imply that the parametric model is robust and sufficient for estimation.
Due to the potential heterogeneity in the treatment group, the \ac{drdid} + DL estimator is arguably the most robust model.
One can see that the \ac{drdid} + DL estimator is still able to retrieve significant results as the parametric models, even though the standard errors are slightly larger.
The magnitude of the \ac{att} is also larger than in the other models, implying an even bigger effect of the compensation policy in Kentucky than previously assumed.

Even though the \ac{drdid} + DL is more robust to unspecified models and heterogeneity in the treatment group, there is the potential issue of overfitting.
Note that the neural network used in Table \ref{tab:reg_results} has the same architecture as in the simulation study.
Contrary to the simulation study, the neural network reports in this application a higher validation loss ($0.467$) than training loss ($0.445$).
Even though the size of the difference is small, it is a hint towards overfitting.
Arguably the issue of heterogenous treatment effects is more severe than the issue of overfitting.

Finally, the results of the \ac{drdid} + DL estimator are quite promising and emphasize more applications of deep learning in observational studies.
In settings with large $n$, conditional \ac{pta}, and potential heterogeneity in the treatment group, the \ac{drdid} + DL estimator seems to be a robust and efficient alternative.

\section{Discussion and Further Research}

%write that section first if I know if I write the application
%cross sectional data extension
%what is the problem with cross sectional data
Naturally, this thesis cannot do justice to all aspects of semiparametric estimation with deep learning, such that there are multiple ways to extend the results in further research.
One possible extension is to apply the aforementioned methods on repeated cross-sectional data.
Altough panel data is the preferred data structure for causal inference due to its generally lower variance, cross-sectional data is often the only available option.
Nonetheless, \citet{santannaDoublyRobustDifferenceindifferences2020} and \citet{manfeDifferenceInDifferenceDesignRepeated} demonstrate the usefulness of implementing \ac{did} on repeated cross-sectional data.
Applying deep learning to this data structure could be a promising approach to estimate causal effects.

%santana and callaway how to implement it for multiple time periods
Another extension is to move the classical 2x2 \ac{did} setting and apply the deep learning approach to multiple time periods or groups.
A natural extension would be the use of the work of \citet{callawayDifferenceinDifferencesMultipleTime2021}, which extend the results of \citet{santannaDoublyRobustDifferenceindifferences2020} that form the basis of this thesis.
The main contributions of  \citet{callawayDifferenceinDifferencesMultipleTime2021} are the application of the \ac{or}, \ac{ipw}, and \ac{drdid} methods to multiple time periods and groups.
Accounting for conditonal \ac{pta} and heterogenous treatment effect in this setting would make the application of deep learning particularly interesting.

The work of \citet{dechaisemartinDifferenceinDifferencesEstimatorsIntertemporal2024} would be another interesting extension to apply deep learning to.
Similar to \citet{callawayDifferenceinDifferencesMultipleTime2021}, they extend the \ac{did} estimator to multiple time periods and groups but focus on non-binary, non-absorbing treatments with lags.

%discuss the shortcoming of my results
%overfitting on L2 regularization
%which archtiecture to choose
%more understanding of the inherent structure of the data (like in the farrell paper l2 regularization)
A major criticism of the deep learning approach is the arbitrary choice of architecture and hyperparameters.
There is still a lack of understanding of the inherent structure deep learning and how to choose the right architecture.
It is unclear how the number of hidden layers, the number of neurons, or the choice of activation function impacts the results.
\citet{farrellDeepNeuralNetworks2021} discusse the unclear effect of l2 regularization on deep learning in inference, even though often used in practice.
In Section 4.4, I discuss that this thesis, the choice of hyperparameters can influence the magnitude of the results.
More research needs to be done to give guidance on how to choose the right architecture and hyperparameters for deep learning in causal inference.

Even though the effect of the size of the arhitecture is unclear for inference, it is intuitive that more hidden layers and neurons are computationally more expensive.
For economist, this issue has been a minor concern in the past, but with the advent of deep and machine learning, it becomes more important.
Especially for large and complex data sets, the computational cost of deep learning can be demanding and computation time extensive.
Further research is needed to understand how to make deep learning computationally more efficient as suggested by \citet{farrellDeepNeuralNetworks2021}.

%what is new stuff
%%ausblick auf neue deep learning approaches (direct estimation with deepl by farrel paper 2)
A relatively new approach to using deep learning for inference is the direct estimation of treatment effects.
Instead of incorporating deep learning within a semiparametric framework, it is used directly to recover parameter functions, as suggested by the work of \citet{DeepLearningIndividual2021}.
This approach allows for second-stage inference, such as estimating how treatment impacts evolve over time or across different subgroups.
Incorporating direct estimation of \ac{att} with deep learning, rather than using it solely for first-stage estimation, could provide an interesting extension for estimating causal effects.

\section{Conclusion}

In this thesis, I have pres

\clearpage


\pagenumbering{Roman}
\section*{Appendix}

[add picture of the relu function]

\printbibliography



\clearpage
\input{./statement}



% \appendix

% The chngctr package is needed for the following lines.
% \counterwithin{table}{section}
% \counterwithin{figure}{section}

\end{document}
