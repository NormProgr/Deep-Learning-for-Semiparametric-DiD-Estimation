\section{Methodology}

%\subsection{Notation and Setup}



\subsection{2x2 Difference in Differences}
%derive the explanation of 2x2 difference in difference from classic to TWFE and talk about heterogeneous effects
To introduce a common ground for the rest of the thesis I want to introduce the notation for the basic 2x2 \ac{did} model.
The model has two time periods given by $T$, where $t \in 0, 1$. These define the pre- and post-treatment period of the policy change.
Note that I use in the whole thesis panel data such that $i$ denotes the individual over the time periods.
The two groups are defined by $D$, where $d \in 0, 1$, where $d = 1$ is the treatment group and $d = 0$ is the control group.
The outcome variable is given by $Y$ and the variable of interest we investigate, the \ac{att}, is given by $\tau^{fe} = \mathbb{E}(Y_{1,1} - Y_{0,1} \mid  X,D = 1)$.
For my \ac{mcs} I use the following common \ac{twfe} model notation as in \citet{santannaDoublyRobustDifferenceindifferences2020}:
\begin{equation}
Y_{it} = \alpha_1 + \alpha_2 T_i + \alpha_3 D_i + \tau^{fe} (T_i \cdot D_i) + \theta' X_i + \epsilon_{it}
\label{eq:twfe}
\end{equation}
Equation \ref{eq:twfe} implicates two assumptions that are discussed in this thesis.
First, it assumes homogeneity in treatment effects, such that $\tau^{fe}$ is constant over all individuals.
Second, it assumes that the \ac{pta} holds, such that the treatment and control group would have developed similarly in the absence of the policy change such that $\mathbb{E} [Y_1 - Y_0 \mid X, D = d] = \mathbb{E} [Y_1 - Y_0 \mid D = d]$.
If one or both of these assumptions are violated the \ac{twfe} estimator in Eq. \ref{eq:twfe} is inconsistent and biased.

To control for heterogeneous treatment effects and to account for conditional \ac{pta}, we can extend the model in Eq. \ref{eq:twfe} by adding interactions of $X$, $T$, and $D$ \citep[see][]{manfeDifferenceInDifferenceDesignRepeated, hansen2022econometrics}.
In the case if the conditional \ac{pta} holds, we can rewrite the Eq. \ref{eq:twfe} the following:
\begin{equation}
    Y_{it} = \alpha + \gamma T_{it} + \beta D_{i} + \delta (T_{it} \cdot D_{i}) + X_{it}' \theta + (T_{it} \cdot X_{it}') \omega + (D_{i} \cdot X_{it}') \nu + (T_{it} \cdot D_{i} \cdot X_{it}') \rho + \epsilon_{it}
    \label{eq:twfecorr}
\end{equation}
Note that $T_{it} \cdot D_{i} \cdot X_{it}'$ is the change of the treatment depending on X and thus the conditional \ac{pta} holds \citep{manfeDifferenceInDifferenceDesignRepeated}.
Eq. \ref{eq:twfecorr} is therefore, in a correctly specified case, neither biased nor inconsistent.
The issue is that the econometric practitioner needs very good reasoning and understanding to add the right interactions.
In the following sections I introduce more flexible techniques circumventing this issue.
\subsection{Outcome Regression}
I want to emphasize \ac{or} as an technical part of \ac{drdid} than using this approach itself for estimating \ac{att}.
\ac{or} is a generalized \ac{did} approach that estimates the outcomes as a function of covariates.
The basic idea is to use the control group to predict what would have happened to the treated individuals if they had not received the treatment.
The prediction can be done trough a linear regression but also other non-linear models like p-nearest neighbor matching \parencite{heckmanMatchingEconometricEvaluation1998}.


\subsection{Inverse Probability Weighting}


\subsection{Double Robust Difference in Differences}
