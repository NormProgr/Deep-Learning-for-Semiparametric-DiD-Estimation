\section{Application}

An early application of \ac{did} is the paper of \citet{meyer1990workers} who investigate the effect of a workers' compensation reform on their time out of work.
In 1980, the states of Kentucky and Michigan substantially increased the compensations in case of work-induced disability or injury.
As the policy affected high-earning workers, \citet{meyer1990workers} took low-earning workers as a control group.
Their idea was that low- and high-earning workers are comparable except that high-earning workers are treated with the compensation policy.
The distribution of the pre-treatment out-of-work duration for low- and high-earning workers can be seen in the Appendix Figure \ref{fig:log_duration_distribution}.
In their original study, they report a significant increase in time out of work for Kentucky but not for Michigan.

\citet{meyer1990workers} implemented a classical $2 \times 2$ \ac{did} design, which makes it suitable to the methods discussed in this thesis.
Due to the low sample size of the Michigan data, the analysis is solely focused on Kentucky.
Therefore, the \ac{did} identification strategy can be formulated as follows:
\begin{equation}
\text{Duration}_{it} = \alpha + \beta_1 \text{Post}_t + \beta_2 \text{HighEarnings}_i + \beta_3 (\text{Post}_t \times \text{HighEarnings}_i) + \gamma X_{it} + \epsilon_{it},
\label{eq:duration}
\end{equation}
where the interaction $(\text{Post}_t \times \text{HighEarnings}_i)$ is the \ac{did} estimator.
$X_{it}$ is a vector of control variables such as injury type, age, or gender.
As \citet{meyer1990workers} use many of these pre-treatment covariates like age or gender it implicates that they assume conditional \ac{pta}.
By design, it is not possible to test for \ac{pta} but the conditional \ac{pta} seems to be a more robust assumption in an observational study \citep{santannaDoublyRobustDifferenceindifferences2020}.
A second remark is towards heterogeneity in the treatment group, which is given in almost all contexts \citep{DeepLearningIndividual2021}.
The Appendix Table \ref{tab:duration} shows the difference in duration of out-of-work time across injury types before and after the treatment.
The magnitude of the differences is quite large, hinting towards heterogeneity in the treatment group.
Based on that, I conducted a regression exclusion test following \citet{hansen2022econometrics} where the results can be seen in Table \ref{tab:exclusion_test} in the Appendix.
The regression exclusion test is significant, which implies that homogeneous treatment effects cannot be assumed.

These results are reason to apply the \ac{drdid} + DL\footnote[5]{Here I use the prebuilt software implementation of \citet*{doubleml2024R} to be able to report summary statistics for the \ac{drdid} + DL estimation.} estimator to the data of \citet{meyer1990workers}.
To compare different designs, I also estimate a saturated dummy design without controls and the regression equation \ref{eq:duration} with controls and interactions.
The estimates are reported in Table \ref{tab:reg_results}.
Note that the results of the regression equation \ref{eq:duration} are slightly different from the results of \citet{meyer1990workers}.
This should be due to different handling of the data-cleaning process than to the different estimation methods.

\begin{table}[ht]
\centering
\begin{threeparttable}
\caption{Regression Results}
\label{tab:reg_results}
\begin{tabular}{lcccccc}
\toprule
\hline
\addlinespace
Model & Coef. & Std.Err. & t & P>|t| & [0.025 & 0.975] \\
\midrule
Saturated design & 0.191 & 0.069 & 2.782 & 0.005 & 0.056 & 0.325 \\
Regression Eq. (14)  & 0.172 & 0.064 & 2.694 & 0.007 & 0.047 & 0.297 \\
\ac{drdid} + DL & 0.250 & 0.075 & 3.34 & 0.000 & 0.103 & 0.396\\
Authors' model & 0.162 & 0.059 &  2.745 & 0.006 & 0.046 & 0.278 \\
\bottomrule
\end{tabular}
\begin{tablenotes}
    \item Notes: In this table are reported the results of a saturated dummy design without controls, the regression equation (14) with controls and interactions, the \ac{drdid} with deep learning and controls and the results from \citet{meyer1990workers} of Table 6 Column (ii). Reported are the coefficients as the \ac{att}, the standard errors, the t-values, P>|t| is the p-value, and the lower- and upper bound of the 95 percent confidence interval. The dependent variable is the log of the duration of work leave. The dataset is taken from the online resources of \citet{wooldridge2019introductory}. The sample size is $n = 5347$.
\end{tablenotes}
\end{threeparttable}
\end{table}


In Table \ref{tab:reg_results}, one can see that the results for all parametric models are quite similar.
Note that the results of the saturated design are similar to the regression equation \ref{eq:duration} implying that the controls might not have a large impact on the results.
Assuming a homogenous treatment effect this result would normally imply that the parametric model is robust and sufficient for estimation.
Due to the potential heterogeneity in the treatment group, the \ac{drdid} + DL estimator is arguably the most robust model.
One can see that the \ac{drdid} + DL estimator is still able to retrieve significant results as the parametric models, even though the standard errors are slightly larger.
The magnitude of the \ac{att} is also larger than in the other models, implying an even bigger effect of the compensation policy in Kentucky than previously assumed.

Even though the \ac{drdid} + DL is more robust to unspecified models and heterogeneity in the treatment group, there is the potential issue of overfitting.
Note that the neural network used in Table \ref{tab:reg_results} has the same architecture as in the simulation study.
Contrary to the simulation study, the neural network reports in this application a higher validation loss ($0.467$) than training loss ($0.445$).
Even though the size of the difference is small, it is a hint towards overfitting.
Arguably the issue of heterogenous treatment effects is more severe than the issue of overfitting.

Finally, the results of the \ac{drdid} + DL estimator are quite promising and emphasize more applications of deep learning in observational studies.
In settings with large $n$, conditional \ac{pta}, and potential heterogeneity in the treatment group, the \ac{drdid} + DL estimator seems to be a robust and efficient alternative.
