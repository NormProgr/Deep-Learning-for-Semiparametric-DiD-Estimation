\section{Conclusion}

In this thesis, I have revised common semiparametric \ac{did} estimators and implemented deep learning as a novel approach to estimate the \ac{att}.
In the setting of conditional \ac{pta} and panel data, I have compared the \ac{or}, \ac{ipw}, and \ac{drdid} first step estimation approaches.
The results show that the \ac{drdid} estimator outperforms the other estimators in terms of bias and variance under homogenous treatment effect.
The deep learning implementation of the \ac{drdid} estimator achieves near as good results as the \ac{drdid} estimator.
Under heterogenous treatment effects, the \ac{drdid} with deep learning outperforms all other comparable estimation strategies.
I illustrate the methods by conducting a simulation study and applying them to an empirical data set.
The results of the Kentucky workers compensation data suggests that the \ac{drdid} estimator with deep learning is a promising approach to estimate the \ac{att}.

The results underline the potential of deep learning in causal inference, especially in cases of complex and large data.
Especially under strong heterogenous treatment effects deep learning seems to be an advisable approach.
A shortcoming is the arbitrary choice of the deep learning architecture and the choice of suitable settings for deep learning.
Further research is needed to apply deep learning for \ac{did} estimation on repeated cross-sectional data or on multiple time and group period cases.
