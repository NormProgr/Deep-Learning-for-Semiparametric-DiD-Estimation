\section{Conclusion}

In this thesis, I have revised common semiparametric \ac{did} estimators and implemented deep learning as a novel approach to estimate the \ac{att}.
In the context of conditional \ac{pta} and panel data, I have compared the \ac{or}, \ac{ipw}, and \ac{drdid} first-step estimation approaches.
The results indicate that the \ac{drdid} estimator outperforms the other estimators in terms of bias and variance under homogenous treatment effect.
The deep learning implementation of the \ac{drdid} estimator achieves nearly as good results as the traditional \ac{drdid} estimator.
Under heterogeneous treatment effects, the \ac{drdid} with deep learning outperforms all other comparable estimators.
To illustrate these methods, I conducted a simulation study and applied the methods to an empirical data set.
The results of Kentucky's worker's compensation data suggest that the \ac{drdid} estimator with deep learning is a promising approach to estimate the \ac{att} under heterogenous treatment effects.

The results underline the potential of deep learning in causal inference, especially in cases of complex and large data.
Particularly under strong heterogeneous treatment effects, deep learning seems to be an advisable approach.
However, the shortcoming is the arbitrary choice of the deep learning architecture and suitable settings.
Further research is needed to apply deep learning for \ac{did} estimation on repeated cross-sectional data or on multiple time and group period cases.
